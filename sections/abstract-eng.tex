\begin{titlepage}
\begin{center}

{\large \textbf{DATA DRIVEN GYMNASTICS SKILLS RECOGNITION AND ANALYSIS}}

\vspace*{1cm}

{\Large \textbf{Abstract}}

\vspace*{1cm}

\end{center}


The present thesis's primary goal is to construct a dataset consisting of gymnastics activities - the backflip and the back handspring, use pose estimation, implement pre-processing strategies, and develop machine learning models capable of distinguishing between the two gymnastics activities.

Backflips and back handsprings are two of the foundational gymnastics moves performed from a standing position and including the rotation of the human body, making it challenging for machine learning algorithms designed to recognize more straightforward human actions to recognize these more complex moves.

The majority of solutions available for estimating poses in gymnastics use movement-restricting and invasive motion capture tools. Other forms of estimating poses from a distance include multi-camera setups, which are not usable in daily gymnasium environments. This thesis aims to explore and propose a more accessible alternative to non-invasive gymnastics skill recognition solutions. The solution presented in this thesis uses a consumer-grade video recorder with a single-camera pose estimation to extract time-series skeleton data. After successfully extracting skeleton data and applying data processing strategies, machine learning classifiers are developed using recurrent neural networks to train the machine to make successful distinctions between backflips and back handsprings.

Current research's primary outcome is classifier models, capable of differentiating backflip and back handspring activities, providing prediction performance up to 95\%. Discussion and classifier analysis section of this thesis gives more in-depth insights into the artificial neuron activations by visualization.

The present thesis is written in English and is \pageref{LastPage} pages long, including 7 chapters, 1 table and 24 figures.

\end{titlepage}


