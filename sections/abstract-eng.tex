\begin{titlepage}
\begin{center}

{\large \textbf{Analysis of Interpretable Anomalies and Kinematic Parameters in Luria's Alternating Series Tests for Parkinson's Disease Modeling}}

\vspace*{1cm}

{\Large \textbf{Abstract}}

\vspace*{1cm}

\end{center}


Primary goal of present thesis is to conduct analysis of patterns drawn during Luria’s alternating series tests, extract interpretable feature set and develop machine learning model, capable of correct differentiation of Parkinson’s disease patients from healthy control subjects.

Luria’s alternating series fine motor tests are being used in psychology and neurology to assess level of disorder in motion planning and execution during handwriting, which is approved biomarker for Parkinson’s disease.

A novel method to analyze Luria’s alternating series patterns drawn during fine motor test constitute main result of the present thesis. Majority of solutions available in the literature are based either on the analysis of entire drawing or individual strokes. Distinctive feature of the proposed approach is that it allows to analyze patterns, considering their logical structure with any required level of detail. To achieve this, unique supervised and unsupervised machine learning techniques are applied. Computer vision technique is used to split pattern into logical segments. Based on this information, feature sequences describing different kinematic properties of the drawing are constructed. During next stages, neural-network based models are used to generate feature sequences of the "expected" normal drawing, which allows to expose "unexpected" regions with anomalies. 

Main outcome of current research is classifier model, capable of differentiating Parkinson's disease patients from healthy controls, providing prediction performance around 91\%. Experimental part of the thesis offers technique for explaining individual predictions of obtained classifier by applying recently proposed machine learning meta-algorithm.

Present thesis is written in English and is \pageref{LastPage} pages long, including 11 chapters, 16 tables and 19 figures.


\end{titlepage}


