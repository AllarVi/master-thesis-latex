In short summarization, the contributions made by the author in this thesis include:

\begin{easylist}[itemize]

& A balanced dataset with backflip and back handspring samples, acquired using a consumer-grade action camera and extracting poses with OpenPose.
& Development of data pre-processing strategies relevant for skeleton-based pose estimation and applicable for standing backflips and back handsprings - activities including human body rotations.
& Recurrent neural network classifiers capable of making successful distinctions between backflips and back handsprings.
& Time-series skeleton-based recurrent neural network output activation interception and visualization for interpreting the outcomes of classifiers. 
& Combining two different research fields, pose estimation and human action recognition, to achieve a full prototype capable of recognizing two gymnastics activities.

\end{easylist}

This thesis's primary purpose was to give an overview of a full end-to-end solution of combining recent advancements in pose estimation with human action recognition in the sports of gymnastics. To the best of the author's knowledge, there is not much literature demonstrating such solutions for gymnastics action recognition. Most previous solutions capturing gymnastics actions use restricting devices or garments. Restricting solutions have not found use in daily training sessions with complex biomechanical movements, such as in the sports of gymnastics. The main requirements set by the author were completed. These included using a consumer-grade action camera (\textit{GoPro Hero 7}) to capture back handsprings and backflips in regular gymnastics training sessions, with no additional equipment. The author demonstrates how open sourced pose estimation tool \textit{OpenPose} could be used as a viable option for capturing the movements. Finally, classifiers with recurrent network architecture were developed and integrated with skeleton-based time-series data as an input. This thesis also includes a demonstration of neuron activations usable with time-series data, uncommon in relevant literature processed by the author. As the author himself is a regular practitioner of this sport, the motivation for experimenting with non-invasive and accessible technology exists. 

Although an end-to-end solution is described in this paper, there is much room for improvement to bring this solution into everyday use by athletes. It should also be noted, that most of the computational work was done in a cloud computing environment with professional tier graphical processing units. Before field testing this solution, the data pre-processing strategies developed and used in this thesis also require optimization. The data pre-processing strategies were needed to balance the shortcomings of \textit{OpenPose}. The constructed dataset should also be extended with more relevant actions and angles of samples for better reusability.

Advancements in the accessible and non-invasive pose estimation and human action recognition could bring the technology closer to be used for feedback in live sports events, automated training session documentations, and relevant metrics acquisition from a distance.










