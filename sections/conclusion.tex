\begin{comment}

    - Analyze full battery, higher accuracy
    - Build regressor model, PD severity



% The main purpose of this thesis was 
% The main goal of this thesis was


% Overall results of the thesis show that problems stated were addressed and goals successfully achieved, but they leave room for improvement. 

% Possible future studies should lie in the area of improving the employed methodology – other possible methods to be evaluated,

% - Full battery
% Possible aspects and future improvement ideas may include:


\end{comment}


Primary goal of present thesis was to conduct analysis of patterns drawn during Luria’s alternating series tests, extract interpretable feature set and develop machine learning model, capable of correct differentiation of Parkinson’s disease patients from healthy controls. Additional goal was to offer solution for explaining individual predictions of obtained classifier.

Research is based on handwriting data collected from 17 patients with diagnosed Parkinson's disease and 17 healthy control subjects within same age group. 

Novel clustering technique was applied during analysis. Proposed clustering algorithm is computer vision based and capable of processing periodic drawing patterns with relatively positioned and scaled elements, preserving logical structure with arbitrary level of detail. Obtained concept could also be applied in other research fields. 

Anomaly detection concept was also introduced and applied. Anomaly detection technique evolved from clustering concept. Method is neural network based and capable of identifying "unseen" or "unexpected" sequence segments within drawing patterns.

Detailed statistical analysis of extracted features was successfully conducted and revealed high number of significant kinematic parameters, derived from certain logical segments of the pattern. Main outcome of the analysis was machine learning classifier, capable of differentiation Parkinson’s disease patients from healthy controls, providing average prediction performance around 91\%. 

% Detailed analysis of patterns drawn during Luria's alternating series tests was successfully conducted. Main outcome of the analysis was machine learning classifier, capable of differentiation Parkinson’s disease patients from healthy controls, providing average prediction performance around 91\%. 

Technique for explaining individual predictions of obtained classifier based on ”Local Interpretable Model-Agnostic Explanations” algorithm was successfully integrated into final solution and performed adequately.

Proposed technique could be included into decision support framework for Parkinson's disease screening and, in theory, adopted by clinicians in medical facilities. Obtained methodology can certainly help to reveal hidden relations between pattern instance parameters and particular classification result with some degree of confidence. With such classification reasoning, clinicians are capable of making informed decisions about whether to trust the model’s predictions.

\vspace{0.5cm}

Obtained results of present thesis clearly indicate, that main goals were successfully achieved. Present research can evolve in different ways. 

\vspace{0.5cm}

Most obvious research direction --- extension of present machine learning binary classifier to regression model, capable of predicting severity of Parkinson's disease, expressed in unified Parkinson's disease rating scale \textit{UPDRS}. Without doubt, it will require a reasonably large group of patients with various  levels of disease severity. 

Additionally, interesting and promising anomaly detection concept could be investigated much further. Novel feature engineering, parameter tuning, experiments with neural network architecture --- are some of possible improvement areas.

% We conducted full and detailed analysis of patterns drawn Luria's alternating series test. Dta

% All goals of the 

% Full and detailed analysis of the patterns was implemented and is complex multilevel process. Novel clustering solution was proposed, 
% anomaly detection concept was derived from clustering,  
% We achieved goals: novel clustering algorithm 

% Primary achievement of thesis --- classifier with 91\% accuracy, which makes it relatively high among present literature.