\section{Visualizing recurrent network activations with time-series skeleton data}
\label{visualizing-recurrent-network-appendix}

\begin{lstlisting}[language=Python]
def get_activations(model, example_sample): 
    n_timesteps, n_features = example_sample.shape[0], example_sample.shape[1]
    
    x = np.zeros((1, n_timesteps, n_features))
    
    for t, timestep in enumerate(example_sample):
        for f, feature in enumerate(timestep):
            x[0, t, f] = example_sample[t][f]
                
    output = model.get_layer('simple_rnn_1').output
    
    f = K.function([model.input], [output])
    
    return f([x])[0][0]

activations = get_activations(example_model, example_sample)

def get_image(img, n_timesteps, img_idx, cell_size=48):
    img_width = n_timesteps * 25
    cell_size = int(img_width / n_timesteps)
    
    pil_image = PILImage.fromarray(img.astype(np.uint8))
    
    resized_pil_image = pil_image.resize((img_width, cell_size))
    
    draw = ImageDraw.Draw(resized_pil_image)
    
    for n_timestep in range(n_timesteps):
        text = str((img_idx * 30) + n_timestep)
        xy = (n_timestep * cell_size, 0)
        
        draw.text(xy, text)
        
    f = BytesIO()
    resized_pil_image.save(f, 'png')
    return Image(data=f.getvalue())

def visualize_neurons(act, cell_size=48):
    n_neurons = act.shape[1]
    n_timesteps = act.shape[0]
    
    fill_value = 128
    
    img = np.full((n_neurons + 1, n_timesteps, 3), fill_value)
    
    # add 1 to each value in matrix and then divide by 2
    scores = (act[:, :].T + 1) / 2
    
    img[1:, :, 0] = 255 * (1 - scores)
    img[1:, :, 1] = 255 * scores

    first_hs_img = img[:, :30, :]
    second_hs_img = img[:, 30:60, :]
    third_hs_img = img[:, 60:90, :]
    fourth_hs_img = img[:, 90:, :]
    
    imgs = [first_hs_img,
            second_hs_img,
            third_hs_img,
            fourth_hs_img]
    
    actual_imgs = []
    for i, img in enumerate(imgs):
        n_img_timesteps = img.shape[1]
        
        actual_imgs.append(get_image(img, n_img_timesteps, i))
    
    return actual_imgs

example_sample_imgs = visualize_neurons(activations)

for img in example_sample_imgs:
    display(img)
\end{lstlisting}