\section{Problem statement}

The following list depicts the sub-problems of the problem statement:

\begin{easylist}[itemize]

    & \textit{Acquiring and exploring relevant data} --- This thesis examines the data needed for capturing complex biomechanical movements without being invasive towards the athletes. From the author's own experience in practicing gymnastics, many already developed restricting motion capture solutions do not apply to the everyday sports environment. 

    Recording human actions can be broadly classified into two categories: (a) \textbf{optical methods} --- for example the \textit{marker-based} systems used in entertainment, but also newer \textit{marker-less} and \textit{deep learning} based systems \cite{Elhayek_2015_CVPR} and (b) \textit{non-optical methods} --- both single, such as accelerometers or gyroscopes and also complex unit sensing devices, such as inertial/magnetic measurements units \cite{7567551}. Here the author narrows the solutions down to optical methods, excluding non-optical methods regarding the context of gymnastics. For gymnastics, the freedom of movement is crucial, since many of the foundational movements (i.e., round-offs - the movement athletes use to transform horizontal velocity to vertical velocity) require multiple meters of free space and no restricting devices on the body.

    These restrictions rule out wired, movement-restricting, and additional weight requiring non-optical sensors. Furthermore, we also rule out special equipment requiring optical marker-based systems. For example, in the article \cite{Burgess2001KINEMATICAO}, research was conducted for comparing the technique of beginner and advanced athletes. While a fascinating study, the markers were attached only to one side of the athlete's body, ignoring the discrepancies of body sides and complicating the experiment's repeatability for everyday use. The author initiates another requirement in the current paper, assuming that the gymnastics athletes are not interested in wearing special equipment for the daily training sessions.

    The author uses a consumer-grade action camera to capture individual athletes' backflips and back handsprings in regular gymnastics facilities, explained in more detail in section \ref{recording-human-actions}.
    
    & \textit{Pose estimation as a viable option for gymnastics movements} --- The author experiments and develops strategies for acquiring, processing, and augmenting data obtained by newer \textit{pose estimation} software (\textit{OpenPose}). We construct a generalized and reusable dataset for experimenting and developing classifiers capable of differentiating gymnastics movements. Further explanation on the pose estimation process is available in section \ref{pose-estimation-process-and-results}.
    
    & \textit{Human action recognizing classifier development} --- Advances in deep learning accessibility inspire the author to explore human action recognition in the context of gymnastics. We use a previously constructed dataset in conjunction with deep learning network design to develop classifiers capable of differentiating between gymnastics movements. The author also adds a cloud-based training environment description to give details of a fully prototyped end-to-end solution, section \ref{classifier-development}.

    & \textit{Visualizing classifier decisions} --- The need for an interactive, easily usable, and understandable software for coaches and athletes to analyze their training sessions is expressed in the article \cite{doi:10.1080/14763140608522878}, emphasizing the ease of usability and allowance of individual input. While giving athletes the feedback for improving their technique is not in the scope of this thesis, the author still makes an effort to visualize and interpret the decisions of the recognitions made by deep learning classifiers in section \ref{classifier-visualization}.

\end{easylist}
