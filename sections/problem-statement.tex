\section{Problem statement}

\begin{comment}

    With respect to all previously mentioned aspects, it is possible to define scope of the problem current thesis addresses: complete and detailed analysis of patterns drawn during Luria’s alternating series tests, extract interpretable feature set and develop machine learning model, capable of correct differentiation between groups of healthy controls and Parkinson’s disease patients.
    
    
    
    
    
    Present thesis offers solution to aforementioned problems. 
    
    Primary goal of present thesis is to conduct detailed analysis of patterns drawn during Luria’s alternating series tests, extract interpretable feature set and develop machine learning model, capable of correct differentiation between groups of healthy controls and Parkinson’s disease patients.
    
    Main objective of current thesis is to introduce human-understandable novel anomaly-type features for "Luria’s alternating series tests", build machine learning model capable of accurate differentiation between Parkinson's disease patients and also method providing traceability of each individual prediction by applying LIME algorithm \cite{ribeiro2016should}, which should transform untrustworthy classification model into a trustworthy one.
    
    We will combine different ML models to filter outliers, split raw data into meaningful clusters using computer vision algorithms, analyze and extract features of each cluster to detect drawing mistakes, build composite classification models using random forest and neural networks and finally will interpret each individual prediction while following main steps:
    
    % The main gap with current approaches, is what they cannot give any understandable feedback to medical personnel. It is surely possible to create binary classifier with numeric feature set. For example we can use random forest or neural network to complete such task. The model will detect unhealthy individuals with some acceptable precision rate. However, we cannot clearly evaluate such "black box" models because of their hidden structure.
    
    % Despite widespread adoption, machine learning models remain mostly black boxes. Understanding the reasons behind predictions is, however, quite important in assessing trust, which is fundamental if one plans to take action based on a prediction, or when choosing whether to deploy a new model. Such understanding also provides insights into the model, which can be used to transform an untrustworthy model or prediction into a trustworthy one.
    
    
    \begin{easylist}[itemize]
    
    % & \textbf{Acquiring dataset} - drawings of PD patients and controls from previous research is available, also iOS application for iPad Pro is in active development stage. Recording session in local hospitals are scheduled in nearest future.
    
    % & \textbf{Pre-processing data} - Recorded dataset from iPad tablet consists of $x$, $y$ coordinates, pressure $p$, azimuth and altitude angles of the pencil and time stamp $t$. Coordinate units $x$, $y$ and pressure $p$ are device-specific and don't give adequate characteristics of the drawing, therefore we need to convert all data records to physical units. Removing outliers, data noise along with obvious faulty input will be filtered \cite{lee2008trajectory}.
    
    % & \textbf{Defining scope of mistakes} - there is a wide range of possibilities, and we should choose certain mistake types to implement. Potential set of mistakes will be: Geometry distortion-type drawing mistakes, skewness of drawing/segment, decline/increase in the amplitude of pattern, angle of upper and lower boundaries of the drawing/segment, rotation distortion, completeness of drawing/segment, anomalies in time series.
    
    % & \textbf{Building of the models}
    % && During implementation of geometry distortion-type mistakes detection algorithm, common trigonometry rules and Euclidean distances would be used, along with common techniques of clustering such as $K$-means, density-based scanning and hierarchical clustering. Some concepts and inspiration would be taken from relevant research paper \cite{hammond2011recognizing}.
    
    % && During implementation of computed kinematic-based and pressure-based mistakes detection model, similar algorithm described in research paper \cite{nomm2016recognition}, would be reproduced and improved. Model would base on combination of popular ML algorithms, such as $K$-means clustering, $k$-nearest neighbors and random forest RF.
    
    & \textbf{Assessment of statistical significance} - following question will be answered: do generated features provide statistically significant differentiation between groups of healthy individuals and Parkinson's disease patients?
    
    & \textbf{Composite model building and validation} - complex decision making Machine Learning algorithm will be developed based on all possible and acquired mistake-type features, kinematic and pressure parameters. To overcome and even benefit from high dimensionality training data, \textit{XGBoost  Scalable Tree Boosting} \cite{chen2016xgboost} algorithm will be applied during model building process.
    
    & \textbf{Interpretation of the models} - "Local Interpretable Model-Agnostic Explanations" LIME algorithm \cite{ribeiro2016should} will be used to describe each individual prediction and SP-LIME algorithm will be applied, revealing hidden relations between features inside our classifier.
    
    \end{easylist}

\end{comment}
