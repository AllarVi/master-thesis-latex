\begin{comment}

% researches suggest\cite{drotar2016evaluation, drotar2015decision, rosenblum2013handwriting}, that handwriting and drawing might serve as a diagnostic biomarker for PD. 

% Main goal of our work is to detect and analyze drawing mistakes within Luria’s alternating series tests of drawing patterns \cite{luria1995higher}, distinguish most significant features, build classification machine learning model and offer mechanism to interpret each individual prediction.

\end{comment}

% About
% - Parkinson's
% - hard to diagnose
% - handwriting biomarker
% - luria

Introduction...

% tendency showing movement in direction of different tests digitized versions



% extend previous research by introducing novel method of drawing pattern analysis 

% Distinction from others
% - luria patterns are rarely used
% - existing studies only analyze whole pattern or strokes
% - logical structure of the pattern is not being taken into account
% - new clustering solution, relatively positioned and sized pattern elements
% - pattern transformation into tree-like graph structures
% - interpretable clusters complimented with parameters
% - anomaly detection 

% Thesis organized as follows
% - 
% - 
% - 


% Main focus of present thesis is to analyze Luria’s drawing patterns of tested individuals, extract interpretable features and produce machine learning model, capable of correct differentiation between groups of healthy controls and Parkinson’s disease patients.

% Luria’s alternating series fine motor tests are being used in psychology and neurology to assess level of disorder in motion planning and execution during handwriting, which is approved biomarker for Parkinson’s disease diagnosis.

% A novel method to analyze Luria’s alternating series patterns drawn during fine motor test constitute main result of the present thesis. Majority of solutions available in the literature are based either on the analysis of entire drawing or individual strokes. Distinctive feature of the proposed approach is that it allows to analyze patterns considering their logical structure with any required level of detail. To achieve this, unique supervised and unsupervised machine learning techniques are applied. Computer vision technique is used to split pattern into logical segments. Based on this information, feature sequences describing different kinematic properties of the drawing are constructed. During next stages, neural-network based models are used to generate feature sequences of the "expected" normal drawing, which allows to highlight "unexpected" regions with anomalies.