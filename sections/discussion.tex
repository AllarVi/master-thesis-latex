
\begin{comment}

Possible improvements:

Explaining individual predictions. A model predicts that a patient has PD, and LIME highlights the features extracted from pattern with significance measures and actual value, that led to the prediction. With these, clinician is capable of making informed decision about whether to trust the model’s prediction.


% The problems addressed in the current thesis were supported by conducting testing with real patients, but small amount of the subjects makes this a pilot research.

% Not many handwriting features yielded the desired distinctiveness, but a usable subset was discovered. 

% The analysis of the results proved that currently employed method has its flaws.

% The quantitative analysis employed for the purpose of the current work was limited to statistical method of evaluation,

% At the moment, it is obvious that currently used methodology is viable and is worth to be investigated further

% The testing results of the current study showed that the compared populations (PDs and healthy controls) most likely to have certain similarities

Phrases: 

% improvement ideas
% possible drawbacks
% And this work clearly indicates that
% should be continued
% As it was mentioned in the section
% Another limitation of presented implementation
% Yet another drawback of current implementation
% As it was also discussed, ...
% there are several drawbacks of using ...
% It should be possible to greatly improve stroke classification
% One of the main challenges in using the machine learning...
% Even though... is not completely finished and has a lot of open issues, it is still can be used 

% % % % % % % % % % % % % % % % % % % % % % % % % % % 

% Results were confirmed by Fisher score and two-sample t-test.
% Overall we are pretty happy with significance of the extracted features. Edge related, kinematic type features derived from logical segments of the pattern performed best. It was validated by statistical analysis and confirmed with performance of generated classifier. 
% Geometry features along with pressure, altitude, latitude angle of stylus demonstrated low significance. Best non-kinematic feature pressure nc also received high Fisher score. These features stand for total number of changes in pressure during drawing test, and can, in theory, represent tremor — one of the symptoms of Parkinson’s disease.

% By achieving highest average classification accuracy around 91\% for Random Forest model, trained with $n=30$ top features, we accomplished primary thesis goal.

\end{comment}

\section{Future work}

Even though this thesis successfully demonstrates an end-to-end solution for gymnastics action classification, it is not an out-of-the-box readily usable solution and has a lot of potential for improvements. Following, is a brief discussion about some ideas for future work:

\begin{easylist}[enumerate]

& \textit{Real-time recognition with action classification} --- The potential of current work includes packaging the code and deploying it as a backend service. For this, some restructuring of the code is necessary. Also, two main services should be developed for better contextual boundaries. The first service should deal with the pose estimation and pre-processing of the data. For pose estimation, the advice would be to deploy OpenPose on a server with at least one graphics processing unit and the data-preprocessing strategies have been thoroughly described in section \ref{data-pre-processing-strategies}. The Python based technology stack used for data pre-processing in this thesis was used mainly for demonstration and validation purposes. The second proposed backend service would be responsible for the classification task of gymnastics action recognition. After initial model training, the model should be deployed as backend service with prediction API usable by a controller tier application. It should also be feasible to implement the continuous learning of the model. For complete real-time solution, a controller tier application should be developed, aggregating the two main backend services described before. The controller tier application should be deployed onto a device with video recording capabilities, i.e. a smart phone.

& \textit{Multi-person action recognition} ---

    
\end{easylist}


